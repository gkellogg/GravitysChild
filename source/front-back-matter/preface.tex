%%%%%%%%%%%%%%%%%%%%%%preface.tex%%%%%%%%%%%%%%%%%%%%%%%%%%%
% 
% Preface
% 
%%%%%%%%%%%%%%%%%%%%%%%%%%%%%%%%%%%%%%%%%%%%%%%%%%%%%%%

\chapter*{Preface}
\addcontentsline{toc}{chapter}{Preface} % Add to TOC

Every book I have ever read has had typos, plot inconsistencies, mathematical errors, or conceptual errors. The best books keep these to a minimum, but they are still present. Can't we do better? The Internet should allow us to give feedback to authors and publishers, so they can fix their work, and make improvements.

Software developers have been doing this for a long time. GitHub is a site that allows many people to work on a single software project. In the case of Open Source Software, where the software is licensed for use by anyone, GitHub repositories of the source code may be copied (``forked'', in the parlance), modified by someone else, and the changes hopefully submitted to the original developers via a so-called ``pull request''. A contributor asks the original developers to ``pull'' his or her suggested changes into the project's source code.

This book is an experiment in collaborative writing. The book's source material is provided on GitHub, just like an Open Source Software project, and is licensed under the GNU Free Documentation License, Version 1.3. Anyone can fork the book into their own GitHub project, make suggestions for changes, and submit a pull request to me. If I accept some or all of your suggestions, I will add your name to the acknowledgements, and thank you publicly.

This book is hosted on GitHub at \url{https://github.com/prototypo/MovingMars}. I can also be reached via \url{https://hyland-wood.org}.

Come, let us write together.


% TODO: Typeset properly. Fredericksburg, Virginia, United States of America, June 2015


% If the chapter ends in an odd page, you may want to skip having the page
%  number in the empty page
\newpage
\thispagestyle{empty}
